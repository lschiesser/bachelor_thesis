\documentclass[12pt, twoside]{report}
\usepackage[utf8]{inputenc}
\usepackage{graphicx}
\usepackage[a4paper,width=150mm,top=25mm,bottom=25mm,bindingoffset=6mm]{geometry}
%\usepackage{biblatex}
\usepackage{setspace}
\usepackage[english]{babel} 
\usepackage{stmaryrd}
%\usepackage{natbib}
\usepackage{csquotes}
\usepackage[style=alphabetic, backend=biber]{biblatex}
%\DeclareLanguageMapping{american}{american-apa}
%\usepackage{apacite}
\addbibresource{references.bib}
%\bibliographystyle{plainnat}
%\setcitestyle{authoryear,open={(},close={)}}
\usepackage{tabularx}
\usepackage{xspace}
\usepackage{xcolor}
% packages for reading results
\usepackage{pgfplotstable}
\usepackage{csvsimple}
\usepackage{siunitx}
\usepackage{lscape}

% define col width
\newcolumntype{Y}{>{\hsize=4\hsize}X}
\newcolumntype{s}{>{\hsize=0.25\hsize}X}

\definecolor{Red}{RGB}{255,0,0}
\definecolor{Green}{RGB}{10,200,100}
\definecolor{Blue}{RGB}{10,100,200}
\definecolor{Orange}{RGB}{255,153,0}
\definecolor{Purple}{RGB}{139,0,139}

\newcommand{\denote}[1]{\mbox{ $[\![ #1 ]\!]$}}
\newcommand*\diff{\mathop{}\!\mathrm{d}}
\newcommand{\red}[1]{\textcolor{Red}{#1}}  
\newcommand{\mh}[1]{\textcolor{Blue}{[mht: #1]}}  
\newcommand{\mf}[1]{\textcolor{Orange}{[rl: #1]}}  
\newcommand{\js}[1]{\textcolor{Green}{[js: #1]}} 
\newcommand{\pt}[1]{\textcolor{Purple}{[pt: #1]}} 

% define functions for reading results from csv
\newcommand{\datafoldername}{R4Tex}

% the following code defines the convenience functions
% as described in the main text below

% rlgetvalue returns whatever is the in cell of the CSV file
% be it string or number; it does not format anything
\newcommand{\rlgetvalue}[4]{\csvreader[filter strcmp={\mykey}{#3},
	late after line = {{,}\ }, late after last line = {{}}]
	{\datafoldername/#1}{#2=\mykey,#4=\myvalue}{\myvalue}}

% rlgetvariable is a shortcut for a specific CSV file (myvars.csv) in which
% individual variables that do not belong to a larger chunk can be stored
\newcommand{\rlgetvariable}[2]{\csvreader[]{\datafoldername/#1}{#2=\myvar}{\myvar}\xspace}

% rlnum format a decimal number
\newcommand{\rlnum}[2]{\num[output-decimal-marker={.},
	exponent-product = \cdot,
	round-mode=places,
	round-precision=#2,
	group-digits=false]{#1}}

\newcommand{\rlnumsci}[2]{\num[output-decimal-marker={.},
	scientific-notation = true,
	exponent-product = \cdot,
	round-mode=places,
	round-precision=#2,
	group-digits=false]{#1}}

\newcommand{\rlgetnum}[5]{\csvreader[filter strcmp={\mykey}{#3},
	late after line = {{,}\ }, late after last line = {{}}]
	{\datafoldername/#1}{#2=\mykey,#4=\myvalue}{\rlnum{\myvalue}{#5}}}

\newcommand{\rlgetnumsci}[5]{\csvreader[filter strcmp={\mykey}{#3},
	late after line = {{,}\ }, late after last line = {{}}]
	{\datafoldername/#1}{#2=\mykey,#4=\myvalue}{\rlnumsci{\myvalue}{#5}}}

% MH's command
\newcommand{\brmresults}[2]{\(\beta = \rlgetnum{#1}{Rowname}{#2}{Estimate}{3}\) (\rlgetnum{#1}{Rowname}{#2}{l.95..CI}{3}, \rlgetnum{#1}{Rowname}{#2}{u.95..CI}{3})}
%\brmresults{expt1_brm.csv}{condition}

\graphicspath{ {images/} }

\linespread{1.25}


\begin{document}


\chapter*{Abstract}


\chapter*{Acknowledgements}
I want to thank...

\tableofcontents
\listoffigures
\listoftables
\chapter{Introduction}
\label{chapter01}
\begin{changemargin}{50pt}{50pt}
Introduction for this thesis by explaining motivation for this topic and giving a short overview.
    \\
    Motivation: overwhelmed health care systems, shortages of testing supplies, AI as supplementary technique to help.
\end{changemargin}

\chapter*{Declaration}
I declare that..

\appendix
\chapter{Appendix}	
%\begin{table}
\centering
\begin{tabular}{ll}
Feature                          & Data Type              \\ \hline
Gender                           & Categorical            \\
Age                              & Numerical (discrete)   \\
WBC (White blood cell count)     & Numerical (continuous) \\
Platelets                        & Numerical (continuous) \\
Neutrophils                      & Numerical (continuous) \\
Lymphocytes                      & Numerical (continuous) \\
Monocytes                        & Numerical (continuous) \\
Eosinophils                      & Numerical (continuous) \\
Basophils                        & Numerical (continuous) \\
CRP (C-reactive protein)         & Numerical (continuous) \\
AST (aspartate aminotransferase) & Numerical (continuous) \\
ALT (alanine aminotransferase)   & Numerical (continuous) \\
ALP (alkaline phosphatase)       & Numerical (continuous) \\
GGT (gamma glutamyl transferase) & Numerical (continuous) \\
LDH (lactate dehydrogenase)      & Numerical (continuous) \\
SWAB                             & Categorical           
\end{tabular}
\caption{Overview over all features of the data set}
\label{tab:overview-features}
\end{table}
\begin{table}
\centering
\begin{tabular}{lllll}
Feature                          & Unit                    & Mean   & Std    & 
Median \\ \hline
Age                              & Years                   & 61.33  & 18.05  & 
64     \\
White Blood Cell Count (WBC)     & $10^9$/L & 8.49   & 4.89   & 
7.10   \\
Platelets                        & $10^9$/L & 224.91  & 102.61  & 
204.00 \\
Neutrophils                      & $10^9$/L & 4.64   & 4.50   & 
3.90   \\
Lymphocytes                      & $10^9$/L & 0.88   & 0.87   & 
0.80   \\
Monocytes                        & $10^9$/L & 0.45   & 0.44   & 
0.40   \\
Eosinophils                      & $10^9$/L & 0.04   & 0.12   & 
0.00   \\
Basophils                        & $10^9$/L & 0.01   & 0.03   & 
0.00   \\
C-reactive protein (CRP)         & mg/L                    & 88.93  & 94.32  & 
53.10  \\
Aspartate Aminotransferase (AST) & U/L                     & 53.81  & 57.59  & 
36.00  \\
Alanine Aminotransferase (ALT)   & U/L                     & 42.82  & 45.43  & 
30.00  \\
Alkaline Phosphatase (ALP)       & U/L                     & 42.21  & 75.71  & 
68.00  \\
Gamma Glutamyl Transferase (GGT) & U/L                     & 40.20  & 101.29 & 
0.00  \\
Lactate dehydrogenase (LDH)      & U/L                     & 264.54 & 238.53 
& 254.00
\end{tabular}
\caption{Descriptive statistics for numerical features in data set (including 
missing values as in \cite{RN127})}
\label{tab:feature-dist-NAN}
\end{table}
\begin{table}
\centering
\begin{tabular}{ll}
Feature                          & Number of NaN (in \%) \\ \hline
Gender                           & 0 (0 \%)              \\
Age                              & 2 (0.72 \%)           \\
WBC (White blood cell count)     & 2 (0.72 \%)           \\
Platelets                        & 2 (0.72 \%)           \\
Neutrophils                      & 70 (25.09 \%)         \\
Lymphocytes                      & 71 (25.45 \%)         \\
Monocytes                        & 70 (25.09 \%)         \\
Eosinophils                      & 70 (25.09 \%)         \\
Basophils                        & 71 (25.45 \%)         \\
CRP (C-reactive protein)         & 6 (2.15 \%)           \\
AST (aspartate aminotransferase) & 2 (0.72 \%)           \\
ALT (alanine aminotransferase)   & 13 (4.66 \%)          \\
ALP (alkaline phosphatase)       & 148 (53.05 \%)        \\
GGT (gamma glutamyl transferase) & 143 (51.25 \%)        \\
LDH (lactate dehydrogenase)      & 85 (30.47 \%)         \\
SWAB                             & 0 (0 \%)             
\end{tabular}
\caption{}
\label{tab:nan-overview}
\end{table}


\printbibliography
%\bibliography{references}
\end{document}
