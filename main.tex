\documentclass[12pt, twoside]{report}
\usepackage[utf8]{inputenc}
\usepackage{graphicx}
\usepackage[a4paper,width=150mm,top=25mm,bottom=25mm,bindingoffset=6mm]{geometry}
%\usepackage{biblatex}
\usepackage{setspace}
\usepackage[english]{babel} 
\usepackage{stmaryrd}
%\usepackage{natbib}
\usepackage{csquotes}
\usepackage[style=alphabetic, backend=biber]{biblatex}
%\DeclareLanguageMapping{american}{american-apa}
%\usepackage{apacite}
\addbibresource{references.bib}
%\bibliographystyle{plainnat}
%\setcitestyle{authoryear,open={(},close={)}}
\usepackage{tabularx}
\usepackage{xspace}
\usepackage{xcolor}
% packages for reading results
\usepackage{pgfplotstable}
\usepackage{csvsimple}
\usepackage{siunitx}
\usepackage{lscape}

% define col width
\newcolumntype{Y}{>{\hsize=4\hsize}X}
\newcolumntype{s}{>{\hsize=0.25\hsize}X}

\definecolor{Red}{RGB}{255,0,0}
\definecolor{Green}{RGB}{10,200,100}
\definecolor{Blue}{RGB}{10,100,200}
\definecolor{Orange}{RGB}{255,153,0}
\definecolor{Purple}{RGB}{139,0,139}

\newcommand{\denote}[1]{\mbox{ $[\![ #1 ]\!]$}}
\newcommand*\diff{\mathop{}\!\mathrm{d}}
\newcommand{\red}[1]{\textcolor{Red}{#1}}  
\newcommand{\mh}[1]{\textcolor{Blue}{[mht: #1]}}  
\newcommand{\mf}[1]{\textcolor{Orange}{[rl: #1]}}  
\newcommand{\js}[1]{\textcolor{Green}{[js: #1]}} 
\newcommand{\pt}[1]{\textcolor{Purple}{[pt: #1]}} 

% define functions for reading results from csv
\newcommand{\datafoldername}{R4Tex}

% the following code defines the convenience functions
% as described in the main text below

% rlgetvalue returns whatever is the in cell of the CSV file
% be it string or number; it does not format anything
\newcommand{\rlgetvalue}[4]{\csvreader[filter strcmp={\mykey}{#3},
	late after line = {{,}\ }, late after last line = {{}}]
	{\datafoldername/#1}{#2=\mykey,#4=\myvalue}{\myvalue}}

% rlgetvariable is a shortcut for a specific CSV file (myvars.csv) in which
% individual variables that do not belong to a larger chunk can be stored
\newcommand{\rlgetvariable}[2]{\csvreader[]{\datafoldername/#1}{#2=\myvar}{\myvar}\xspace}

% rlnum format a decimal number
\newcommand{\rlnum}[2]{\num[output-decimal-marker={.},
	exponent-product = \cdot,
	round-mode=places,
	round-precision=#2,
	group-digits=false]{#1}}

\newcommand{\rlnumsci}[2]{\num[output-decimal-marker={.},
	scientific-notation = true,
	exponent-product = \cdot,
	round-mode=places,
	round-precision=#2,
	group-digits=false]{#1}}

\newcommand{\rlgetnum}[5]{\csvreader[filter strcmp={\mykey}{#3},
	late after line = {{,}\ }, late after last line = {{}}]
	{\datafoldername/#1}{#2=\mykey,#4=\myvalue}{\rlnum{\myvalue}{#5}}}

\newcommand{\rlgetnumsci}[5]{\csvreader[filter strcmp={\mykey}{#3},
	late after line = {{,}\ }, late after last line = {{}}]
	{\datafoldername/#1}{#2=\mykey,#4=\myvalue}{\rlnumsci{\myvalue}{#5}}}

% MH's command
\newcommand{\brmresults}[2]{\(\beta = \rlgetnum{#1}{Rowname}{#2}{Estimate}{3}\) (\rlgetnum{#1}{Rowname}{#2}{l.95..CI}{3}, \rlgetnum{#1}{Rowname}{#2}{u.95..CI}{3})}
%\brmresults{expt1_brm.csv}{condition}

\graphicspath{ {images/} }

\linespread{1.25}


\begin{document}


\chapter*{Abstract}


\chapter*{Acknowledgements}
I want to thank...

\tableofcontents
\listoffigures
\listoftables
\chapter{Introduction}
\label{chapter01}
\begin{changemargin}{50pt}{50pt}
Introduction for this thesis by explaining motivation for this topic and giving a short overview.
    \\
    Motivation: overwhelmed health care systems, shortages of testing supplies, AI as supplementary technique to help.
\end{changemargin}

\chapter*{Declaration}
I declare that..

\appendix
\chapter{Appendix}	
%\begin{table}
\centering
\begin{tabular}{ll}
Feature                          & Data Type              \\ \hline
Gender                           & Categorical            \\
Age                              & Numerical (discrete)   \\
WBC (White blood cell count)     & Numerical (continuous) \\
Platelets                        & Numerical (continuous) \\
Neutrophils                      & Numerical (continuous) \\
Lymphocytes                      & Numerical (continuous) \\
Monocytes                        & Numerical (continuous) \\
Eosinophils                      & Numerical (continuous) \\
Basophils                        & Numerical (continuous) \\
CRP (C-reactive protein)         & Numerical (continuous) \\
AST (aspartate aminotransferase) & Numerical (continuous) \\
ALT (alanine aminotransferase)   & Numerical (continuous) \\
ALP (alkaline phosphatase)       & Numerical (continuous) \\
GGT (gamma glutamyl transferase) & Numerical (continuous) \\
LDH (lactate dehydrogenase)      & Numerical (continuous) \\
SWAB                             & Categorical           
\end{tabular}
\caption{Overview of all features of the data set}
\label{tab:overview-features}
\end{table}
\begin{table}
\centering
\begin{tabular}{lllll}
Feature                          & Unit                    & Mean   & Std    & 
Median \\ \hline
Age                              & Years                   & 61.33  & 18.05  & 
64     \\
White Blood Cell Count (WBC)     & $10^9$/L & 8.49   & 4.89   & 
7.10   \\
Platelets                        & $10^9$/L & 224.91  & 102.61  & 
204.00 \\
Neutrophils                      & $10^9$/L & 4.64   & 4.50   & 
3.90   \\
Lymphocytes                      & $10^9$/L & 0.88   & 0.87   & 
0.80   \\
Monocytes                        & $10^9$/L & 0.45   & 0.44   & 
0.40   \\
Eosinophils                      & $10^9$/L & 0.04   & 0.12   & 
0.00   \\
Basophils                        & $10^9$/L & 0.01   & 0.03   & 
0.00   \\
C-reactive protein (CRP)         & mg/L                    & 88.93  & 94.32  & 
53.10  \\
Aspartate Aminotransferase (AST) & U/L                     & 53.81  & 57.59  & 
36.00  \\
Alanine Aminotransferase (ALT)   & U/L                     & 42.82  & 45.43  & 
30.00  \\
Alkaline Phosphatase (ALP)       & U/L                     & 42.21  & 75.71  & 
68.00  \\
Gamma Glutamyl Transferase (GGT) & U/L                     & 40.20  & 101.29 & 
0.00  \\
Lactate dehydrogenase (LDH)      & U/L                     & 264.54 & 238.53 
& 254.00
\end{tabular}
\caption{Descriptive statistics for numerical features in data set (including 
missing values as in \cite{RN127})}
\label{tab:feature-dist-NAN}
\end{table}
\begin{table}
\centering
\begin{tabular}{ll}
Feature                          & Number of NaN (in \%) \\ \hline
Gender                           & 0 (0 \%)              \\
Age                              & 2 (0.72 \%)           \\
WBC (White blood cell count)     & 2 (0.72 \%)           \\
Platelets                        & 2 (0.72 \%)           \\
Neutrophils                      & 70 (25.09 \%)         \\
Lymphocytes                      & 71 (25.45 \%)         \\
Monocytes                        & 70 (25.09 \%)         \\
Eosinophils                      & 70 (25.09 \%)         \\
Basophils                        & 71 (25.45 \%)         \\
CRP (C-reactive protein)         & 6 (2.15 \%)           \\
AST (aspartate aminotransferase) & 2 (0.72 \%)           \\
ALT (alanine aminotransferase)   & 13 (4.66 \%)          \\
ALP (alkaline phosphatase)       & 148 (53.05 \%)        \\
GGT (gamma glutamyl transferase) & 143 (51.25 \%)        \\
LDH (lactate dehydrogenase)      & 85 (30.47 \%)         \\
SWAB                             & 0 (0 \%)             
\end{tabular}
\caption{Number of missing values and their proportion of the total number of 
data 
points}
\label{tab:nan-overview}
\end{table}
\begin{table}
\centering
\begin{tabular}{lllll}
Feature                          & Unit                    & Mean   & Std    & 
Median \\ \hline
Age                              & Years                   & 61.78  & 17.81  & 
64     \\
White Blood Cell Count (WBC)     & $10^9$/L & 8.55   & 4.86   & 
7.10   \\
Platelets                        & $10^9$/L & 226.5  & 101.2  & 
205.00 \\
Neutrophils                      & $10^9$/L & 6.20   & 4.17   & 
5.10   \\
Lymphocytes                      & $10^9$/L & 1.19   & 0.80   & 
1.00   \\
Monocytes                        & $10^9$/L & 0.61   & 0.41   & 
0.50   \\
Eosinophils                      & $10^9$/L & 0.06   & 0.13   & 
0.00   \\
Basophils                        & $10^9$/L & 0.01   & 0.04   & 
0.00   \\
C-reactive protein (CRP)         & mg/L                    & 90.89  & 94.42  & 
54.20  \\
Aspartate Aminotransferase (AST) & U/L                     & 54.20  & 57.61  & 
36.00  \\
Alanine Aminotransferase (ALT)   & U/L                     & 44.92  & 45.50  & 
31.00  \\
Alkaline Phosphatase (ALP)       & U/L                     & 89.89  & 89.09  & 
71.00  \\
Gamma Glutamyl Transferase (GGT) & U/L                     & 82.48  & 132.70 & 
41.00  \\
Lactate dehydrogenase (LDH)      & U/L                     & 380.45 & 193.98 & 
328.00
\end{tabular}
\caption{Descriptive statistics for numerical features in data set (excluding 
missing values)}
\label{tab:feature-dist}
\end{table}
% non-normality of data with graphs from python as visualization aid


\begin{table}
\begin{tabular}{lll}
Feature                          & Test statistic & p-value \\ \hline
Age                              & 0.976          & 0.000   \\
WBC (White blood cell count)     & 0.873          & 0.000   \\
Platelets                        & 0.930          & 0.000   \\
Neutrophils                      & 0.838          & 0.000   \\
Lymphocytes                      & 0.785          & 0.000   \\
Monocytes                        & 0.811          & 0.000   \\
Eosinophils                      & 0.457          & 0.000   \\
Basophils                        & 0.395          & 0.000   \\
CRP (C-reactive protein)         & 0.836          & 0.000   \\
AST (aspartate aminotransferase) & 0.556          & 0.000   \\
ALT (alanine aminotransferase)   & 0.629          & 0.000   \\
ALP (alkaline phosphatase)       & 0.420          & 0.000   \\
GGT (gamma glutamyl transferase) & 0.500          & 0.000   \\
LDH (lactate dehydrogenase)      & 0.877          & 0.000  
\end{tabular}
\caption{Results of Shapiro-Wilk test for normality for numerical features of 
the data set}
\label{tab:shapiro-wilk}
\end{table}

\begin{table}
\centering
\begin{tabular}{lllllllllllll}
 &
  \multicolumn{2}{c}{Fold1} &
  \multicolumn{2}{c}{Fold2} &
  \multicolumn{2}{c}{Fold3} &
  \multicolumn{2}{c}{Fold4} &
  \multicolumn{2}{c}{Fold5} &
  \multicolumn{2}{c}{Combined} \\ \hline
                    & mean & std  & mean & std  & mean & std  & mean & std  & 
mean & std  & mean & std  \\ \hline
RF       & 0.72 & 0.01 & 0.84 & 0.03 & 0.81 & 0.03 & 0.75 & 0.03 & 
0.77 & 0.03 & 0.78 & 0.05 \\
LR & 0.71 & 0.03 & 0.81 & 0.02 & 0.74 & 0.03 & 0.70 & 0.06 & 
0.73 & 0.04 & 0.74 & 0.05
\end{tabular}
\caption{Mean training accuracies and their standard deviations (std) 
for each training fold and all training folds combined. Random Forest = RF, 
Logisitic Regression = LR }
\label{tab:training-acc}
\end{table}

\begin{table}
\centering
\begin{tabular}{lllllllllllll}
 &
  \multicolumn{2}{c}{Fold1} &
  \multicolumn{2}{c}{Fold2} &
  \multicolumn{2}{c}{Fold3} &
  \multicolumn{2}{c}{Fold4} &
  \multicolumn{2}{c}{Fold5} &
  \multicolumn{2}{c}{Combined} \\ \hline
   & mean & std  & mean & std  & mean & std  & mean & std  & mean & std  & mean 
& std  \\ \hline
RF & 0.69 & 0.01 & 0.81 & 0.04 & 0.79 & 0.03 & 0.73 & 0.03 & 0.71 & 0.04 & 0.75 
& 0.05 \\
LR & 0.67 & 0.02 & 0.78 & 0.03 & 0.72 & 0.03 & 0.68 & 0.06 & 0.67 & 0.04 & 0.70 
& 0.06
\end{tabular}
\caption{Mean training balanced accuracies and their standard deviations (std) 
for each training fold and all training folds combined. Random Forest = RF, 
Logisitic Regression = LR }
\label{tab:training-b_acc}
\end{table}
\begin{table}
\centering
\begin{tabular}{lrrrrrrr}
Model &   acc &  b\_acc &  sensitivity &  specificity &   ppv &   auc &  
a\_precision \\ \hline
   RF & 0.768 &  0.737 &        0.909 &        0.565 & 0.750 & 0.802 &        
0.829 \\
   LR & 0.732 &  0.694 &        0.909 &        0.478 & 0.714 & 0.808 &        
0.855 \\
   DT & 0.714 &  0.685 &        0.848 &        0.522 & 0.718 & 0.685 &        
0.698 \\
\end{tabular}
\caption{Validation metrics for all classifiers; RF = Random Forest, DT = 
Decision Tree, LR = Logisitic Regression}
\label{tab:validation-metrics}
\end{table}

\begin{table}[ht]
\centering
\begin{adjustbox}{width=\textwidth}
\begin{tabular}{llllrrrrrr}
Fold & Imputed Set & Model & Model Count &   Accuracy &  Balanced Accuracy &  
Sensitivity &  
Specificity & PPV & AUC \\ \hline
   0 &           1 &    LR &           0 & 0.750 &  0.709 &        0.939 &      
 
 0.478 & 0.721 & 0.816 \\
   0 &           2 &    LR &           1 & 0.714 &  0.679 &        0.879 &      
 
 0.478 & 0.707 & 0.837 \\
   0 &           3 &    LR &           2 & 0.696 &  0.663 &        0.848 &      
 
 0.478 & 0.700 & 0.806 \\
   0 &           4 &    LR &           3 & 0.696 &  0.663 &        0.848 &      
 
 0.478 & 0.700 & 0.797 \\
   0 &           5 &    LR &           4 & 0.679 &  0.642 &        0.848 &      
 
 0.435 & 0.683 & 0.826 \\
   1 &           1 &    LR &           5 & 0.804 &  0.787 &        0.825 &      
 
 0.750 & 0.892 & 0.845 \\
   1 &           2 &    LR &           6 & 0.839 &  0.831 &        0.850 &      
 
 0.812 & 0.919 & 0.872 \\
   1 &           3 &    LR &           7 & 0.804 &  0.787 &        0.825 &      
 
 0.750 & 0.892 & 0.856 \\
   1 &           4 &    LR &           8 & 0.786 &  0.738 &        0.850 &      
 
 0.625 & 0.850 & 0.822 \\
   1 &           5 &    LR &           9 & 0.804 &  0.769 &        0.850 &      
 
 0.688 & 0.872 & 0.833 \\
   2 &           1 &    LR &          10 & 0.732 &  0.713 &        0.818 &      
 
 0.609 & 0.750 & 0.742 \\
   2 &           2 &    LR &          11 & 0.768 &  0.750 &        0.848 &      
 
 0.652 & 0.778 & 0.764 \\
   2 &           3 &    LR &          12 & 0.768 &  0.757 &        0.818 &      
 
 0.696 & 0.794 & 0.823 \\
   2 &           4 &    LR &          13 & 0.732 &  0.720 &        0.788 &      
 
 0.652 & 0.765 & 0.773 \\
   2 &           5 &    LR &          14 & 0.696 &  0.683 &        0.758 &      
 
 0.609 & 0.735 & 0.758 \\
   3 &           1 &    LR &          15 & 0.714 &  0.695 &        0.771 &      
 
 0.619 & 0.771 & 0.774 \\
   3 &           2 &    LR &          16 & 0.750 &  0.733 &        0.800 &      
 
 0.667 & 0.800 & 0.788 \\
   3 &           3 &    LR &          17 & 0.643 &  0.619 &        0.714 &      
 
 0.524 & 0.714 & 0.686 \\
   3 &           4 &    LR &          18 & 0.750 &  0.733 &        0.800 &      
 
 0.667 & 0.800 & 0.777 \\
   3 &           5 &    LR &          19 & 0.625 &  0.605 &        0.686 &      
 
 0.524 & 0.706 & 0.680 \\
   4 &           1 &    LR &          20 & 0.800 &  0.723 &        0.972 &      
 
 0.474 & 0.778 & 0.801 \\
   4 &           2 &    LR &          21 & 0.727 &  0.680 &        0.833 &      
 
 0.526 & 0.769 & 0.781 \\
   4 &           3 &    LR &          22 & 0.709 &  0.629 &        0.889 &      
 
 0.368 & 0.727 & 0.773 \\
   4 &           4 &    LR &          23 & 0.691 &  0.627 &        0.833 &      
 
 0.421 & 0.732 & 0.754 \\
   4 &           5 &    LR &          24 & 0.727 &  0.680 &        0.833 &      
 
 0.526 & 0.769 & 0.785 \\
   0 &           1 &    RF &           0 & 0.732 &  0.694 &        0.909 &      
 
 0.478 & 0.714 & 0.792 \\
   0 &           2 &    RF &           1 & 0.714 &  0.685 &        0.848 &      
 
 0.522 & 0.718 & 0.801 \\
   0 &           3 &    RF &           2 & 0.732 &  0.694 &        0.909 &      
 
 0.478 & 0.714 & 0.782 \\
   0 &           4 &    RF &           3 & 0.732 &  0.700 &        0.879 &      
 
 0.522 & 0.725 & 0.739 \\
   0 &           5 &    RF &           4 & 0.714 &  0.692 &        0.818 &      
 
 0.565 & 0.730 & 0.760 \\
   1 &           1 &    RF &           5 & 0.875 &  0.856 &        0.900 &      
 
 0.812 & 0.923 & 0.848 \\
   1 &           2 &    RF &           6 & 0.821 &  0.800 &        0.850 &      
 
 0.750 & 0.895 & 0.842 \\
   1 &           3 &    RF &           7 & 0.839 &  0.812 &        0.875 &      
 
 0.750 & 0.897 & 0.850 \\
   1 &           4 &    RF &           8 & 0.839 &  0.812 &        0.875 &      
 
 0.750 & 0.897 & 0.809 \\
   1 &           5 &    RF &           9 & 0.804 &  0.750 &        0.875 &      
 
 0.625 & 0.854 & 0.806 \\
   2 &           1 &    RF &          10 & 0.804 &  0.787 &        0.879 &      
 
 0.696 & 0.806 & 0.825 \\
   2 &           2 &    RF &          11 & 0.857 &  0.839 &        0.939 &      
 
 0.739 & 0.838 & 0.839 \\
   2 &           3 &    RF &          12 & 0.786 &  0.772 &        0.848 &      
 
 0.696 & 0.800 & 0.816 \\
   2 &           4 &    RF &          13 & 0.804 &  0.781 &        0.909 &      
 
 0.652 & 0.789 & 0.847 \\
   2 &           5 &    RF &          14 & 0.804 &  0.787 &        0.879 &      
 
 0.696 & 0.806 & 0.831 \\
   3 &           1 &    RF &          15 & 0.750 &  0.714 &        0.857 &      
 
 0.571 & 0.769 & 0.826 \\
   3 &           2 &    RF &          16 & 0.696 &  0.700 &        0.686 &      
 
 0.714 & 0.800 & 0.783 \\
   3 &           3 &    RF &          17 & 0.768 &  0.738 &        0.857 &      
 
 0.619 & 0.789 & 0.848 \\
   3 &           4 &    RF &          18 & 0.786 &  0.762 &        0.857 &      
 
 0.667 & 0.811 & 0.812 \\
   3 &           5 &    RF &          19 & 0.768 &  0.729 &        0.886 &      
 
 0.571 & 0.775 & 0.793 \\
   4 &           1 &    RF &          20 & 0.782 &  0.734 &        0.889 &      
 
 0.579 & 0.800 & 0.836 \\
   4 &           2 &    RF &          21 & 0.764 &  0.695 &        0.917 &      
 
 0.474 & 0.767 & 0.798 \\
   4 &           3 &    RF &          22 & 0.727 &  0.655 &        0.889 &      
 
 0.421 & 0.744 & 0.848 \\
   4 &           4 &    RF &          23 & 0.764 &  0.708 &        0.889 &      
 
 0.526 & 0.780 & 0.798 \\
   4 &           5 &    RF &          24 & 0.800 &  0.760 &        0.889 &      
 
 0.632 & 0.821 & 0.777 \\
\end{tabular}
\end{adjustbox}
\caption{Evaluation metrics for Random Forest (RF) and Logisitc Regression (LR) 
during the nested cross validation}
\label{tab:storage}
\end{table}

\afterpage{%
\newgeometry{width=175mm,top=10mm,bottom=15mm,bindingoffset=6mm}
% material for this page
\begin{sidewaystable}
\begin{adjustbox}{width=\textwidth}
\begin{tabular}{lrrrrrrrrrrrrrrrrrrrrrrrrrrrr}
{} & \multicolumn{2}{c}{AGE} & \multicolumn{2}{c}{WBC} & 
\multicolumn{2}{c}{Platelets} & \multicolumn{2}{c}{Neutrophils} & 
\multicolumn{2}{c}{Lymphocytes} & \multicolumn{2}{c}{Monocytes} & 
\multicolumn{2}{c}{Eosinophils} & \multicolumn{2}{c}{Basophils} & 
\multicolumn{2}{c}{CRP} & \multicolumn{2}{c}{AST} & \multicolumn{2}{c}{ALT} & 
\multicolumn{2}{c}{ALP} & \multicolumn{2}{c}{GGT} & \multicolumn{2}{c}{LDH} \\ 
\hline
{} &   mean &    std &  mean &   std &      mean &     std &        mean &   std 
&        mean &   std &      mean &   std &        mean &   std &      mean &   
std &   mean &    std &   mean &    std &   mean &    std &   mean &    std &   
mean &     std &    mean &     std \\ \hline
original & 61.776 & 17.816 & 8.553 & 4.855 &   226.532 & 101.174 &       6.200 & 
4.173 &       1.187 & 0.806 &     0.606 & 0.410 &       0.055 & 0.132 &     
0.014 & 0.039 & 90.889 & 94.421 & 54.202 & 57.613 & 44.917 & 45.503 & 89.893 & 
89.090 & 82.478 & 132.703 & 380.448 & 193.984 \\
1.0      & 61.839 & 17.773 & 8.574 & 4.863 &   226.654 & 100.820 &       6.597 & 
4.482 &       1.260 & 0.899 &     0.649 & 0.455 &       0.066 & 0.147 &     
0.017 & 0.041 & 92.581 & 96.875 & 54.208 & 57.458 & 47.459 & 49.868 & 82.362 & 
66.091 & 79.681 & 119.072 & 379.516 & 199.693 \\
2.0      & 61.832 & 17.768 & 8.557 & 4.840 &   225.951 & 101.067 &       6.597 & 
4.301 &       1.236 & 0.807 &     0.655 & 0.458 &       0.052 & 0.140 &     
0.015 & 0.039 & 91.247 & 94.973 & 54.269 & 57.422 & 47.405 & 50.648 & 81.803 & 
65.194 & 73.470 & 103.199 & 365.706 & 178.115 \\
3.0      & 61.871 & 17.792 & 8.600 & 4.870 &   226.288 & 100.896 &       6.650 & 
4.439 &       1.220 & 0.801 &     0.659 & 0.473 &       0.054 & 0.123 &     
0.014 & 0.038 & 91.278 & 94.059 & 54.161 & 57.417 & 46.774 & 48.087 & 83.484 & 
67.061 & 75.283 & 106.761 & 367.760 & 191.073 \\
4.0      & 61.728 & 17.768 & 8.557 & 4.839 &   226.571 & 100.911 &       6.625 & 
4.353 &       1.229 & 0.807 &     0.643 & 0.429 &       0.054 & 0.124 &     
0.016 & 0.043 & 91.447 & 94.246 & 54.029 & 57.452 & 46.864 & 48.285 & 83.065 & 
66.928 & 75.384 & 113.305 & 374.771 & 195.309 \\
5.0      & 61.832 & 17.806 & 8.534 & 4.843 &   226.446 & 101.323 &       6.592 & 
4.412 &       1.209 & 0.799 &     0.642 & 0.446 &       0.055 & 0.126 &     
0.015 & 0.039 & 91.255 & 94.812 & 54.004 & 57.453 & 45.943 & 45.804 & 83.964 & 
67.706 & 76.444 & 117.722 & 374.405 & 196.479 \\
6.0      & 61.839 & 17.773 & 8.574 & 4.863 &   226.654 & 100.820 &       6.597 & 
4.482 &       1.260 & 0.899 &     0.649 & 0.455 &       0.066 & 0.147 &     
0.017 & 0.041 & 92.581 & 96.875 & 54.208 & 57.458 & 47.459 & 49.868 & 82.362 & 
66.091 & 79.681 & 119.072 & 379.516 & 199.693 \\
7.0      & 61.832 & 17.768 & 8.557 & 4.840 &   225.951 & 101.067 &       6.597 & 
4.301 &       1.236 & 0.807 &     0.655 & 0.458 &       0.052 & 0.140 &     
0.015 & 0.039 & 91.247 & 94.973 & 54.269 & 57.422 & 47.405 & 50.648 & 81.803 & 
65.194 & 73.470 & 103.199 & 365.706 & 178.115 \\
8.0      & 61.871 & 17.792 & 8.600 & 4.870 &   226.288 & 100.896 &       6.650 & 
4.439 &       1.220 & 0.801 &     0.659 & 0.473 &       0.054 & 0.123 &     
0.014 & 0.038 & 91.278 & 94.059 & 54.161 & 57.417 & 46.774 & 48.087 & 83.484 & 
67.061 & 75.283 & 106.761 & 367.760 & 191.073 \\
9.0      & 61.728 & 17.768 & 8.557 & 4.839 &   226.571 & 100.911 &       6.625 & 
4.353 &       1.229 & 0.807 &     0.643 & 0.429 &       0.054 & 0.124 &     
0.016 & 0.043 & 91.447 & 94.246 & 54.029 & 57.452 & 46.864 & 48.285 & 83.065 & 
66.928 & 75.384 & 113.305 & 374.771 & 195.309 \\
10.0     & 61.832 & 17.806 & 8.534 & 4.843 &   226.446 & 101.323 &       6.592 & 
4.412 &       1.209 & 0.799 &     0.642 & 0.446 &       0.055 & 0.126 &     
0.015 & 0.039 & 91.255 & 94.812 & 54.004 & 57.453 & 45.943 & 45.804 & 83.964 & 
67.706 & 76.444 & 117.722 & 374.405 & 196.479 \\
11.0     & 61.839 & 17.773 & 8.574 & 4.863 &   226.654 & 100.820 &       6.597 & 
4.482 &       1.260 & 0.899 &     0.649 & 0.455 &       0.066 & 0.147 &     
0.017 & 0.041 & 92.581 & 96.875 & 54.208 & 57.458 & 47.459 & 49.868 & 82.362 & 
66.091 & 79.681 & 119.072 & 379.516 & 199.693 \\
12.0     & 61.832 & 17.768 & 8.557 & 4.840 &   225.951 & 101.067 &       6.597 & 
4.301 &       1.236 & 0.807 &     0.655 & 0.458 &       0.052 & 0.140 &     
0.015 & 0.039 & 91.247 & 94.973 & 54.269 & 57.422 & 47.405 & 50.648 & 81.803 & 
65.194 & 73.470 & 103.199 & 365.706 & 178.115 \\
13.0     & 61.871 & 17.792 & 8.600 & 4.870 &   226.288 & 100.896 &       6.650 & 
4.439 &       1.220 & 0.801 &     0.659 & 0.473 &       0.054 & 0.123 &     
0.014 & 0.038 & 91.278 & 94.059 & 54.161 & 57.417 & 46.774 & 48.087 & 83.484 & 
67.061 & 75.283 & 106.761 & 367.760 & 191.073 \\
14.0     & 61.728 & 17.768 & 8.557 & 4.839 &   226.571 & 100.911 &       6.625 & 
4.353 &       1.229 & 0.807 &     0.643 & 0.429 &       0.054 & 0.124 &     
0.016 & 0.043 & 91.447 & 94.246 & 54.029 & 57.452 & 46.864 & 48.285 & 83.065 & 
66.928 & 75.384 & 113.305 & 374.771 & 195.309 \\
15.0     & 61.832 & 17.806 & 8.534 & 4.843 &   226.446 & 101.323 &       6.592 & 
4.412 &       1.209 & 0.799 &     0.642 & 0.446 &       0.055 & 0.126 &     
0.015 & 0.039 & 91.255 & 94.812 & 54.004 & 57.453 & 45.943 & 45.804 & 83.964 & 
67.706 & 76.444 & 117.722 & 374.405 & 196.479 \\
16.0     & 61.839 & 17.773 & 8.574 & 4.863 &   226.654 & 100.820 &       6.597 & 
4.482 &       1.260 & 0.899 &     0.649 & 0.455 &       0.066 & 0.147 &     
0.017 & 0.041 & 92.581 & 96.875 & 54.208 & 57.458 & 47.459 & 49.868 & 82.362 & 
66.091 & 79.681 & 119.072 & 379.516 & 199.693 \\
17.0     & 61.832 & 17.768 & 8.557 & 4.840 &   225.951 & 101.067 &       6.597 & 
4.301 &       1.236 & 0.807 &     0.655 & 0.458 &       0.052 & 0.140 &     
0.015 & 0.039 & 91.247 & 94.973 & 54.269 & 57.422 & 47.405 & 50.648 & 81.803 & 
65.194 & 73.470 & 103.199 & 365.706 & 178.115 \\
18.0     & 61.871 & 17.792 & 8.600 & 4.870 &   226.288 & 100.896 &       6.650 & 
4.439 &       1.220 & 0.801 &     0.659 & 0.473 &       0.054 & 0.123 &     
0.014 & 0.038 & 91.278 & 94.059 & 54.161 & 57.417 & 46.774 & 48.087 & 83.484 & 
67.061 & 75.283 & 106.761 & 367.760 & 191.073 \\
19.0     & 61.728 & 17.768 & 8.557 & 4.839 &   226.571 & 100.911 &       6.625 & 
4.353 &       1.229 & 0.807 &     0.643 & 0.429 &       0.054 & 0.124 &     
0.016 & 0.043 & 91.447 & 94.246 & 54.029 & 57.452 & 46.864 & 48.285 & 83.065 & 
66.928 & 75.384 & 113.305 & 374.771 & 195.309 \\
20.0     & 61.832 & 17.806 & 8.534 & 4.843 &   226.446 & 101.323 &       6.592 & 
4.412 &       1.209 & 0.799 &     0.642 & 0.446 &       0.055 & 0.126 &     
0.015 & 0.039 & 91.255 & 94.812 & 54.004 & 57.453 & 45.943 & 45.804 & 83.964 & 
67.706 & 76.444 & 117.722 & 374.405 & 196.479 \\
21.0     & 61.839 & 17.773 & 8.574 & 4.863 &   226.654 & 100.820 &       6.597 & 
4.482 &       1.260 & 0.899 &     0.649 & 0.455 &       0.066 & 0.147 &     
0.017 & 0.041 & 92.581 & 96.875 & 54.208 & 57.458 & 47.459 & 49.868 & 82.362 & 
66.091 & 79.681 & 119.072 & 379.516 & 199.693 \\
22.0     & 61.832 & 17.768 & 8.557 & 4.840 &   225.951 & 101.067 &       6.597 & 
4.301 &       1.236 & 0.807 &     0.655 & 0.458 &       0.052 & 0.140 &     
0.015 & 0.039 & 91.247 & 94.973 & 54.269 & 57.422 & 47.405 & 50.648 & 81.803 & 
65.194 & 73.470 & 103.199 & 365.706 & 178.115 \\
23.0     & 61.871 & 17.792 & 8.600 & 4.870 &   226.288 & 100.896 &       6.650 & 
4.439 &       1.220 & 0.801 &     0.659 & 0.473 &       0.054 & 0.123 &     
0.014 & 0.038 & 91.278 & 94.059 & 54.161 & 57.417 & 46.774 & 48.087 & 83.484 & 
67.061 & 75.283 & 106.761 & 367.760 & 191.073 \\
24.0     & 61.728 & 17.768 & 8.557 & 4.839 &   226.571 & 100.911 &       6.625 & 
4.353 &       1.229 & 0.807 &     0.643 & 0.429 &       0.054 & 0.124 &     
0.016 & 0.043 & 91.447 & 94.246 & 54.029 & 57.452 & 46.864 & 48.285 & 83.065 & 
66.928 & 75.384 & 113.305 & 374.771 & 195.309 \\
25.0     & 61.832 & 17.806 & 8.534 & 4.843 &   226.446 & 101.323 &       6.592 & 
4.412 &       1.209 & 0.799 &     0.642 & 0.446 &       0.055 & 0.126 &     
0.015 & 0.039 & 91.255 & 94.812 & 54.004 & 57.453 & 45.943 & 45.804 & 83.964 & 
67.706 & 76.444 & 117.722 & 374.405 & 196.479 \\
\end{tabular}
\end{adjustbox}
\caption{Numerical summaries for all variables of the original not-imputed data 
set (denoted by original) and the imputed data sets (denoted by a number)}
\label{tab:imputed}
\end{sidewaystable}
\clearpage
\restoregeometry
}





\printbibliography
%\bibliography{references}
\end{document}
