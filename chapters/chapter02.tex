\section{Data}
The data used to train the classifiers was provided by \citeauthor{RN127} \cite{RN127}. 
It was collected between the end of February 2020 and mid of March 2020 from patients admitted to the \textit{IRCSS Ospedale San Raffaele} and consists of 279 individuals who were selected randomly.
For each individual, the data set provides their age, gender, results of a 
routine blood screening, and the result of a PCR test for Sars-CoV-2.
A complete overview over the recorded features is provided in 
\ref{tab:overview-features}. The target variable \textit{Swab} is binary and 
indicates the result of a PCR-test for Sars-CoV-2 taken by naso-pharyngeal 
swab. A 0 indicates a negative test while a 1 indicates a positive test.
The data set is slightly imbalanced towards positive cases with 102 (37\%) 
negative cases and 177 (63\%) positive cases.
\\
Since the variable \textit{Gender} 
was provided as a string, it was transformed into two binary numerical 
variables called \textit{female} and \textit{male} by one-hot encoding.
Further, I removed two values for the variable \textit{Age}, specifically the 
values 0 and 1. There was no other data recorded from minors under the age of 
18 and therefore these two values can be presumed to be input errors during the 
collection process.
\\
Table \ref{tab:feature-dist} provides common statistics for the numerical 
features of the data set.
% non-normal distribution
As you can see in Figure \ref{fig:density}, most of the data is non-normally 
distributed.
% end with missing values since it's a good transition to MICE
Table \ref{tab:nan-overview} shows that most features have missing values. 196 
samples have at least one feature missing which amounts to 70 \% of the data. 
Due to the small size of data set it is not feasible to exclude these 
individuals from the analysis process. It is rather more constructive to use an 
imputation method that models the missing values based on the observed values in 
the data set. Therefore, \citeauthor{RN127} chose to use \textit{Multivariate 
Imputation by Chained Equations}. 
\section{Multivariate Imputation by Chained Equations}
\textit{Multivariate Imputation by Chained Equations} or \textit{MICE} for 
short is an imputation method proposed by \citeauthor{RN135} \cite{RN135}.
\begin{changemargin}{50pt}{50pt}
Describe MICE algorithm (keine Bewertung, findet in Discussion statt)
\\
Short description of implementation in Python using rpy2
\\
Shortly mention that R implementation is not able to apply MICE model to other 
data only to data it is ``trained''/ ``fitted'' on (or maybe that's for the 
\ref{discussion} Discussion section)
\\
Maybe include PMM?
\end{changemargin}
\section{K-fold nested cross validation}
\section{Classifiers used}
\subsection{Random Forest}
\subsection{Logistic regression}
