In December 2019, a novel coronavirus named Sars-CoV-2 emerged in Wuhan China. 
It spread rapidly around the world, leading the World Health Organization (WHO) 
to declare a global pandemic in March 2020. To date, more than one year 
post-outbreak, the coronavirus disease COVID-19 caused by the Sars-CoV-2 
coronavirus has infected more than 100 million people and has claimed more than 
2 million dead \cite{RN204}. Due to its unprecedented nature and its high 
ability to spread in the general population, it is crucial to identify 
infectious people early and isolate them as soon as possible in order to 
prevent the spread of the virus. Especially the high proportion of asymptomatic 
but infectious patients is a major challenge for the containment of this 
pandemic \cite{RN205}. The main method to identify the virus is the reverse 
polymerase chain reaction (PCR) or reverse transcriptase-PCR (RT-PCR) 
technique. It identifies the virus by detecting the presence of its genetic 
material in a sample taken by nasopharyngeal swab wich collects nasal 
secretion from the back of the nose and throat. Unfortunately, the execution of 
the test is very time-consuming and takes at least four to five hours under 
optimal condition. 
Furthermore, the execution requires the use of special equipment and reagents, 
the involvement of skilled and qualified staff, personnel to collect samples, 
and relies on the proper genetic conservation of the RNA sequences 
\cite{RN201, RN202}.
% irgendwie sollte hier noch ein transition Satz kommen
Medical staff in emergency departments have to assess the probability of a 
Sars-CoV-2 infection for each patient during triage and thereby have to decide 
who should be tested. In the event of a shortage of testing capabilities, this 
can be particularly difficult to do. Algorithmic solutions could provide 
assistance and reduce the uncertainty regarding who should be tested for the 
disease. There are different approaches to this algorithmic solution.
One strategy is based on computer tomography scans or chest X-rays 
\cite{RN200}. Models based on chest X-rays are often associated with high rates 
of false negative predictions \cite{RN200, RN206}. Whereas models based on CT 
scans are not advantageous in comparison with a PCR test as they share common 
requirements like specialized equipment, long duration and high costs.
Another approach is the use of blood tests which are generally available, 
deliver results much faster taking only a minimum of 15 minutes in an emergency 
setting and are much more cost effective.
Therefore, his thesis replicates a feasibility study implementing different 
classifiers to predict the outcome of a PCR test for COVID-19 based on a 
routine blood exam.
% kurzer ausblick über was kommt
