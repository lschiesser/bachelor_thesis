\section{Discussion of Results}
Although the classifiers perform worse than the ones in the original paper, 
most recorded metrics still certify them a good discriminative performance.
With a sensitivity of 90\% for both models and accuracy values between 73\% and 
76\%, the classifiers can still serve as supplementary decision tools to help 
physicians make decisions about allocating testing based on their predictions 
and other indicators. Further, the AUC and AUC (PRC) values which are over 80\% 
for both classifiers indicate that they are indeed strong classifiers.
Due to the slight imbalance of the data set, it is  favorable to examine the 
classifiers regarding this potential source of bias. If there was a significant 
difference between the accuracy and the balanced accuracy for a classifier, it 
would entail a bias against one of the target outcomes. This is not the case 
for both classifiers. Moreover, the high values of the AUC (PRC) suggest that 
the classifiers are not significantly affected by the imbalance of the data.
The metrics can also make statements about an infection event and the 
expressiveness of the classifiers. The high sensitivity signals that 90\% of 
patient who get a positive result actually have the disease. While the 
specificity is comparatively low, it does not necessarily pose a problem. A low 
specificity in combination with a high sensitivity indicates a higher false 
positive rate than a high false negative rate.\cite{RN168}
% rephrase this sentences, it is super weird
Therefore, if the result of the classifier is negative, it will be safe to 
assume that a PCR test for COVID-19 will also be negative and thus the patient 
will not have the disease.
This is more desirable because it is better to be over cautious and recommend 
more tests.
% maybe add PPV part from Schreibplan
\par
% this sentence sould be rewritten
Since the question about the relevance of the classifier has been clarified, 
this paragraph deals with the internal classification reasoning of the 
classifier and their accordance with findings from medical studies.
When inspecting the feature importance plots for the tree-based classifiers, 
both plots (Figure \ref{fig:rf_importance} and \ref{fig:dt_importance}) show 
that the AST, WBC, CRP and lymphocyte counts are under the five most important 
features.
% this sentence is also shit
In the Random Forest, the LDH count replaces the age variable in 
Decision Trees. For the Logistic Regression (Figure \ref{fig:lr_importance}), 
the eosinophil, WBC, basophil and lymphocyte counts are the most important 
features together with the age variable. All feature importance measures have 
the WBC and lymphocyte count in common. Multiple studies also show that these 
two blood values are significant indicators for a COVID-19 infection. The WBC 
or white blood cell count refers to the actual number of white blood cells per 
volume of blood while the lymphocytes are a subgroup of white blood cells 
involved in eliciting a immune response to foreign agents.\cite{RN137, RN188} 
During an infection with Sars-CoV-2, the WBC and lymphocyte counts will 
decrease since this is a sign of a viral infection and the body's response to 
it.\cite{RN162,RN186, RN185}
AST or aspartate aminotransferase, the most important feature in the tree-based 
methods, is an enzyme mainly found in heart and liver, its levels increase when 
the muscles of these organs are injured.\cite{RN189, RN188} This is also the 
case during a COVID-19 infection since the virus attacks not only the upper 
respiratory tracts but the whole body including the liver and 
heart.\cite{RN182} CRP or C-reactive protein is another substance found during 
a COVID-19 infection. It is produced in the liver and is discharged after 
tissue damage, the start of an infection or other inflammatory causes. 
Increased volumes of this protein are often the first indication of an 
infection or inflammation in the body.\cite{RN138, RN188} Elevated levels of 
this protein can also be observed during a COVID-19 infection since it is an 
infection and primarily targets lung tissue.\cite{RN187, RN162} LDH or lactate 
dehydrogenase is an enzyme involved in metabolic cycles for energy production, 
it is present in almost all cells in the body especially in the heart, 
liver, lungs, kidneys, muscles and blood cells. An increase in LDH can indicate 
acute kidney or liver disease, hypoxia, or heart and lung 
infarction.\cite{RN190, RN188} According to \cite{RN162, RN187}, a blood test 
reveals elevated levels of LDH because Sars-CoV-2 primarily attacks the upper 
respiratory tracts which leads to lung damage and the discharge of LDH into the 
blood stream. Age as the fourth important feature in the Decision Tree can also 
be a predictor for a positive result. \cite{RN193} determines that individuals 
older than 70 years are more susceptible to a severe course of the disease or 
even death. Further, the study revealed a higher risk for male individuals.
For eosinophils and basophils, \cite{RN162, RN185} reveal a significant 
decrease for patients with a positive test result. But \cite{RN162} notes that 
these differences might not have any clinical implication during diagnosis since 
the count in healthy individuals is also rather low and exhibits a large 
variability.
% rewrite this
In conclusion, the comparison of the most important features of every 
classifier with the findings from medical research reveals that the models use 
feature that medical research deemed as significant to identify patients with 
COVID-19. Thereby, they can be considered valid in an medical environment.
\par
% Real-world validation
The findings above are all theoretical and have to be proven in the real world.
\begin{changemargin}{50pt}{50pt}
 Plausibility of models in terms of recorded metrics, i.e., sensitivity, 
specificity, ... and what that means for their predictive ``statements''
\\
Accordance of feature importance and scientific research (do most important 
features in model coincide with blood values identified as being evidence for a 
COVID-19 illness)
\end{changemargin}
\section{Discussion of Methods}
% auf non-normality eingehen und die zwei Methoden präsentieren
\begin{changemargin}{50pt}{50pt}
Shortly mention that R implementation is not able to apply MICE model to other 
data only to data it is ``trained''/ ``fitted'' on (or maybe that's for the 
\ref{discussion} Discussion section)
\\
discuss imputation process further and its connection to cross-validation, 
keyword: data leakage
\\
discuss the original paper in light of replication and open science, e.g., lack 
of information about parameters/settings, missing information regarding 
results, ...
\end{changemargin}
\section{Conclusion}
